\documentclass[a4paper,10pt]{article}
%\documentclass[a4paper,10pt]{scrartcl}

\usepackage[utf8]{inputenc}

% ------
% Clickable URLs
\usepackage{hyperref}

\title{PhoneX component list \\ (confidential, subject to NDA)}
\author{Du\v{s}an Klinec}
% \date{23.9.2014}

\pdfinfo{%
  /Title    (PhoneX component list [CONFIDENTIAL])
  /Author   (Du\v{s}an Klinec)
  /Creator  (Du\v{s}an Klinec)
  /Producer ()
  /Subject  ()
  /Keywords ()
}

\begin{document}
\maketitle

\section{PhoneX client application}
Here follows the list of client application components / third party libraries used in PhoneX Android application.

\subsection{PJSIP}\label{sec:pjsip}
\par\smallskip\noindent\textbf{Purpose.} Library used for secure SIP calls in conjunction with ZRTPCPP library (see section \ref{sec:zrtpcpp}). 
Used to establish a direct communication link with a remote party using STUN/TURN/ICE protocols.

\par\smallskip\noindent\textbf{License.} Authors provide dual licensing model. One option is GPL which is undesired for our purpose 
since it requires the whole application to be released under GPL license what implies publishing a source code. 
Another one is a commercial license. Commercial license is royalty free, company based, yearly based.
Intuitively, after buying a license for one year (\pounds 7500 + VAT), the library can be used in unlimited number 
of products (i.e., Android, iOS, desktop versions) for commercial purposes without need to publish a source code, 
no matter how many developers are in the company or how many users is using the application. Detailed license agreement 
is attached in a document package as \verb|PJSIPSoftwareLicenseAgreement1.7.pdf|. GPL license: \\ \url{https://www.gnu.org/licenses/gpl-2.0.html}
\par\smallskip\noindent\textbf{More information.} Written in C/C++. Compatible with Android, iOS, Linux, Windows, MAC OS X. \url{http://www.pjsip.org/}

\subsection{aSMACK}
\par\smallskip\noindent\textbf{Purpose.} Library used for XMPP communication with server side. Currently used for presence and push 
notifications between users and servers. Later messages are planned to be implemented using XMPP protocol.
\par\smallskip\noindent\textbf{License.} Released under Apache License v2 (permissive license, no source code publishing) + OpenLDAP license (permissive license).
Licensing information: \url{https://github.com/Flowdalic/asmack/blob/master/LICENSE}
\par\smallskip\noindent\textbf{More information.} \url{http://www.igniterealtime.org/projects/smack/}

\subsection{kSOAP}
\par\smallskip\noindent\textbf{Purpose.} Library used for SOAP communication with server components. Mainly used in login process and certificate exchange.
\par\smallskip\noindent\textbf{License.} Released under permissive open source license, BSD. \\ \url{https://github.com/mosabua/ksoap2-android/blob/master/LICENSE.txt}
\par\smallskip\noindent\textbf{More information.}  \url{http://kobjects.org/ksoap2/index.html}

\subsection{ZRTPCPP}\label{sec:zrtpcpp}
\par\smallskip\noindent\textbf{Purpose.} Used in conjunction with PJSIP library, implements ZRTP protocol for establishing unique cryptographic keys 
for each SIP voice call.
\par\smallskip\noindent\textbf{License.} Released under LGPL. License allows usage in closed source project under some circumstances. Intuitively, if a LGPL
licensed library is dynamically linked to the closed source application, LGPL is not violated. This use case is permitted and widely used. In PhoneX
we use ZRTPCPP library in this way, i.e., it is dynamically linked to the PJSIP library. LGPL also states that all modifications made to the LGPL licensed 
library have to be published - this is not restrictive for us. LGPL: \url{https://www.gnu.org/licenses/lgpl.html}
\par\smallskip\noindent\textbf{More information.} Written in C/C++. \\ \url{https://github.com/wernerd/ZRTPCPP}

\subsection{webRTC}
\par\smallskip\noindent\textbf{Purpose.} Library used for audio/video processing.
\par\smallskip\noindent\textbf{License.} Released under permissive open source license, BSD. \\ \url{http://www.webrtc.org/license-rights/license}.
\par\smallskip\noindent\textbf{More information.}  \url{http://www.webrtc.org/}

\subsection{Speex codec}
\par\smallskip\noindent\textbf{Purpose.} Main audio codec used for voice encoding.
\par\smallskip\noindent\textbf{License.} Released under permissive open source license, BSD. \\ \url{https://www.xiph.org/licenses/bsd/speex/}.
\par\smallskip\noindent\textbf{More information.}  \url{http://www.speex.org/}

\subsection{Spongy castle}
\par\smallskip\noindent\textbf{Purpose.} Main cryptographic library used for e.g., generating cryptographic keys, certificates, secure certificate storage, 
symmetric and asymmetric encryption. It is an Android port of a BouncyCastle library.
\par\smallskip\noindent\textbf{License.} Released under same permissive open source license as BouncyCastle, addaptation of MIT X11 license. \\ \url{http://www.bouncycastle.org/licence.html}.
\par\smallskip\noindent\textbf{More information.}  SpongyCastle: \url{https://rtyley.github.io/spongycastle/} \quad BouncyCastle: \url{http://www.bouncycastle.org/}

\subsection{SQLCipher}
\par\smallskip\noindent\textbf{Purpose.} Library used for an encrypted storage for PhoneX. We store user profile data, messages and other sensitive information 
to an encrypted SQLCipher database. It provides a transparent encryption layer for SQLite database.
\par\smallskip\noindent\textbf{License.} Released under permissive open source license, BSD. \\ \url{https://www.zetetic.net/sqlcipher/license}.
\par\smallskip\noindent\textbf{More information.}  \url{https://www.zetetic.net/sqlcipher}


\section{PhoneX server side}
Here follows the list of client application components / third party software used on the server side of PhoneX application.

\subsection{MySQL}
\par\smallskip\noindent\textbf{Purpose.} Current database layer used by server side software. 
\par\smallskip\noindent\textbf{License.} Released under GPL license. Poses no restrictions for PhoneX.
Used as provided by standard distribution channels.
\par\smallskip\noindent\textbf{More information.} \url{https://www.mysql.com/}

\subsection{OpenSIPS}
\par\smallskip\noindent\textbf{Purpose.} SIP server used for establishing secure SIP calls. This software is widely deployed in VoIP architectures 
and well tested by community. Has good properties in terms of performance and throughput. It is scalable, capable of using memory caching (memcache).
Works also well in failover/load balancing scenario, a lot of materials available online. 

\par\smallskip\noindent\textbf{License.} Released under GPL license. We made no modifications to the SIP server so this poses no restriction for PhoneX.
We are using it as provided by standard distribution channels.
\par\smallskip\noindent\textbf{More information.} Written in C. \url{http://www.opensips.org/}

\subsection{Openfire}
\par\smallskip\noindent\textbf{Purpose.} XMPP server. Used mainly for user presence and push notifications. 
This software is one of the most common XMPP servers used. Implements a lot of XMPP extensions and provides a good scalability (clustering support).

\par\smallskip\noindent\textbf{License.} Released under permissive open source Apache License v2.
\par\smallskip\noindent\textbf{More information.}  Written in Java, uses embedded Java Tomcat servlet container. \url{http://www.igniterealtime.org/projects/openfire/}

\subsection{Restund}
\par\smallskip\noindent\textbf{Purpose.} Lightweight relay server. Implements TURN/STUN protocol. Relays RTP (audio/video stream) between two 
communicating parties if no direct connection is available.

\par\smallskip\noindent\textbf{License.} Released under permissive open source BSD license.
\par\smallskip\noindent\textbf{More information.}  Written in C. \url{http://www.creytiv.com/restund.html}

\subsection{Certificate server}
\par\smallskip\noindent\textbf{Purpose.} Server used for login, certificate generation / exchange, contact list storage. 
Provides SOAP interface to the PhoneX client.
\par\smallskip\noindent\textbf{License.} Programmed by us for PhoneX.
\par\smallskip\noindent\textbf{More information.}  Written in Java. Runs in Tomcat servlet container. Uses Spring, Hibernate.

\subsection{FileTransfer server}
\par\smallskip\noindent\textbf{Purpose.} Server used for file transfer. Implements our file transfer protocol. 
\par\smallskip\noindent\textbf{License.} Programmed by us for PhoneX.
\par\smallskip\noindent\textbf{More information.}  Written in Java. Runs in Tomcat servlet container. Uses Spring, Hibernate.

\section{Licensing \& cost summary}
PhoneX uses open source software with licenses compatible with commercial application model. Expenses for the whole project 
are only \pounds 7500 + VAT a year for PJSIP commercial license, (see section \ref{sec:pjsip}). PJSIP is compatible with
all target platforms thus no further SIP library is needed.

\end{document}
