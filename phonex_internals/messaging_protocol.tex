\documentclass[a4paper,10pt]{article}
%\documentclass[a4paper,10pt]{scrartcl}

\usepackage[utf8]{inputenc}
\usepackage{amsmath}
\usepackage[letterpaper, margin=1in, left=0.7in]{geometry}
\usepackage{fancyvrb}

% ------
% Clickable URLs
\usepackage{hyperref}

\title{PhoneX messaging protocol draft \\ (confidential, subject to NDA)}
\author{Du\v{s}an Klinec, Miroslav Sv\'{i}tok}
% \date{23.9.2014}

\pdfinfo{%
  /Title    (PhoneX component list [CONFIDENTIAL])
  /Author   (Du\v{s}an Klinec, Miroslav Sv\'{i}tok)
  /Creator  (Du\v{s}an Klinec, Miroslav Sv\'{i}tok)
  /Producer ()
  /Subject  ()
  /Keywords ()
}

\begin{document}
\maketitle

\section{Message protocol v1 - S/MIME}
Message protocol v1 uses S/MIME message format. 
Plaintex message format entering the encryption is: 

\begin{center}
  \verb#s1 = ||1||source-sip||date-of-creation||nonce||text-to-send# 
\end{center}
\begin{itemize}
 \item \verb#1# is the version of the protocol.
 \item \verb#source-sip# denotes creator of the message.
 \item \verb#date-of-creation# denotes date when message was created, i.e., it is a number of milliseconds since Jan. 1, 1970, midnight GMT.
 \item \verb#nonce# is a random integer. Supposed to serve as a protection against a replay attack. Moreover, it helps to detect 
 duplicate messages due to lost ACK packets.
 \item \verb#text-to-send# is a plaintext message entered by user in the application interface.
\end{itemize}

\subsection{Message build algorithm}
\begin{enumerate}
 \item A compression of the $s1$ message to reduce a message size.\\
 \verb#s2 = compress(s1)#
 
 \item Compressed message is signed with the private key of the sender. Scheme used is \verb#SHA1withRSA#. 
 Sender certificate is added to the message so recipient is able to verify the signature. This step is required. Disadvantege is 
 that extra certificate increases size of the sent message.\\
 \verb#s3 = sign(s2, SHA1withRSA)#
 
 \item Signed message is then encrypted with \verb#AES128-CBC#, using an unique symmetric key encrypted with the recipient's public key.\\
 \verb#s4 = encrypt(s3, RSA, AES128-CBC)#
 
 \item Message is base64 encoded because SIP messaging protocol supports only text messages to be transmitted.\\
 \verb#s5 = base64_encode(s4)#
 
\end{enumerate}


\section{Message protocol v2 - based on protocol buffers}

Design of the next messaging protocol was inspired by problems encountered with messaging protocol v1. 
Namely it should solve the following issues:

\begin{itemize}
 \item Allow for a better and modular implementation. Aim for forward compatibility to support future protocol extensions.
 Both secure transport protocol and message protocol can have different versions and separate dispatchers in the implementation
 handling the protocol type.
 \item Message protocol: Add support for a different type of messages, e.g., command messages, push notification messages, ACK messages. Each
 can have a different message format and a different dispatcher handling them.
 \item Reduce the message size: adding a certificate should be an optional step especially in the interactive communication. Attaching a 
 certificate is ineffective in interactive communication. Simple message takes 1kB of data (base64 encoded). Sending such a big message
 with the mobile internet connection proved to be non-reliable due to high packet loss / transmission errors. 
 \item Reduce the message code handling footprint. Currently, {\it BouncyCastle S/MIME} library is used for handling the message 
 processing, i.e., encryption, decryption, signatures, etc... It has non-trivial code footprint and complexity. S/MIME protocol
 might be too complicated for some purposes (i.e., signaling messages) and cannot be used as a general purpose secure transport protocol.
\end{itemize}

\subsection{Message protocol ideas}
\begin{itemize}
 \item The core idea is to clearly separate a {\it secure transport protocol (STP)}, responsible for a secure authenticated message transport to the destination
and a {\it application message protocol (AMP) }, responsible for IM, signaling, etc... The design is inspired by ISO/OSI encapsulation/port multiplexing.
The idea is to have multiple simpler protocols rather than one big, complex and full-fledged protocol.

 \item Both STP and AMP contains version number (like IP), in order to support future protocol extensions.

 \item There may be several distinct STPs defined, each providing different security properties. 
 For example signed STPs provides non-repudiation while STPs featuring deniable authentication
provide privacy. There may be a plain STP providing only authentication, or using only simple symmetric encryption 
using shared secret keys. Plain STP may be used e.g., for ACK messages.

 \item STP is level 1 protocol, AMP is level 2 protocol (AMP is encapsulated in / carried by STP).
 
 \item One of the STP can be S/MIME protocol, already used as v1 message protocol.  
 
 \item Whole message protocol should be independent on the transport protocol used to deliver messages to the endpoints. It can be
 a text-only transport protocol, such as SIP SIMPLE or XMPP message, thus the whole message can be base64 encoded before sending.
\end{itemize}

\subsection{Message protocol design}
\label{subsec:message_protocol_design}
\begin{Verbatim}[frame=single]
message MessageProtocolEnvelope {
  optional int protocolType=1;
  optional int protocolVersion=2;
  optional int sequenceNumber=3; // Optional.
  optional bytes payload=4;
}
\end{Verbatim} 

Payload field contains particular transport protocol. Mapping of the \verb#protocolType# number to the particular type:\\
\begin{Verbatim}[frame=single]
1 = S/MIME, protocol v1.
2 = STPSimple.
3 = Group STP.
4 = STPSimpleAuth. 
4 = Group STPAuth.
// 4 = FileTransfer protocol.
// 5 = OTR protocol.
// 6 = Shared secret key protocol.
// 7 = Plain protocol.
\end{Verbatim} 

\subsection{STPSimple v1, v2}
Lightweight STP, using hybrid encryption, digital signature. Protocol does not provide PFC (perfect forward secrecy).

Application message passed from upper layer (AMP) is denoted as \verb#messagePayload#. 
\begin{Verbatim}[frame=single]
message STPSimple {
  // Application message protocol type (e.g., IM, push, ...)
  // The message is routed to the upper layer dispatcher depending on this number.
  optional int32 ampType=1;
  
  // AMP version number. Supports extensions.
  optional int32 ampVersion=2;
  
  // OPT. Application message sequence number. For ACK purposes, message addressing.
  optional uint32 sequenceNumber=3;
  
  // Timestamp of the message. Replay attack, timestamping.
  optional uint64 messageSentMiliUTC=4;
  
  // Random integer for the message. Replay attack.
  optional uint32 randomNonce=5;
  
  // Sender address identification.
  optional string sender=6;
  
  // Message destination.
  optional string destination=7;
  
  // OPT. Message destination type (address/hash(nonce:address))
  optional int32 destinationType=8;
  
  // Digital signature of the message.
  optional bytes signature=9;
  // OPT.
  optional int32 signatureVersion=10;
  
  // OPT. Certificate version used for the signature.
  optional string certificateVersion=11;
  
  // Initialization vector for symmetric encryption.
  optional bytes iv=12;
  
  // RSA encrypted block. RSA/ECB/OAEPWithSHA1AndMGF1Padding
  optional bytes easymBlock=13;
  // OPT.
  optional int easymBlockVersion=14;
  
  // Symmetric block, ciphertext
  optional bytes esymBlock=15;
  // OPT.
  optional int esymBlockVersion=16;
  
  // HMAC on the message fields
  optional bytes hmac=17;
  // OPT.
  optional int hmacVersion=18;
  
  // Helper fields, used only for signature & HMAC.
  optional bytes encKey=19;
  optional bytes macKey=20;
  optional bytes payload=21;
  
  // Helper fields, used only for MAC computation. 
  optional int32 protocolType=22;
  optional int32 protocolVersion=23;
}
\end{Verbatim} 

\paragraph{Fields discussion.}
\begin{itemize}
 \item \verb#Sequence# number is optional. Intended purpose is to be able to ACKnowledge some messages or answer on some messages.
 \item \verb#randomNonce# is required as an replay attack protection.
 \item \verb#destination# address i.e., the recipient address can be either plaintext, or hashed. By default, \verb#destinationType# can be omitted.
 Default is plaintext recipient address. 
 
 Intended purpose is to hide application
 message destination (can differ from underlying protocol pseudonym) from server or other recipients. Intended recipient can easily verify 
 if the message is addressed for him, others has to perform costly computation. But it is not encryption, attacker still can test the destination.

 \item \verb#signatureVersion# contains signature scheme used for the digital signature. This field can be omitted, default is RSA signature.
 
 \item \verb#signature# is a RSA digital signature protects key message fields and the plaintext message. Signing plaintext 
 is intended to bind plaintext message to the sender identity. Signature binds following fields together:
\begin{itemize}
 \item \verb#protocolType#
 \item \verb#protocolVersion#
 \item \verb#ampType#
 \item \verb#ampVersion#
 \item \verb#sequenceNumber#
 \item \verb#randomNonce#
 \item \verb#sender#
 \item \verb#destination#
 \item \verb#destinationType#
 \item \verb#iv#, a initialization vector for the symmetric encryption.
 \item \verb#encKey01#, a symmetric encryption key for the symmetric encryption block.
 \item \verb#macKey01#, a HMAC symmetric key.
 \item \verb#messagePayload#
\end{itemize}

$
sig = SIGN_{privKey}(hash(protocolType:protocolVersion:ampType:ampVersion:sequenceNumber:randomNonce:sender:
destination:destinationType:iv:encKey01:macKey01:messagePayload)) 
$

\textbf{Update v2}: signature now includes \verb#messageSentMiliUTC#;

\item \verb#iv# is an initialization vector for a symmetric encryption. If the encryption algorithm does not require IV (e.g., ECB. Note: do not use ECB!)
this field is omitted.

\item \verb#easymBlockVersion# contains asymmetric encryption scheme version, field can be omitted, default is \\\verb#RSA/ECB/OAEPWithSHA1AndMGF1Padding#.
\item \verb#easymBlock# contains asymmetrically encrypted symmetric keys. Asymmetric block size depends on the key size, but should be 2048 bits
at least if using RSA encryption, thus several encryption keys fits this block.\\
$easymBlock = AENCRYPT(encKey01:macKey01)$

\item \verb#esymBlockVersion# contains symmetric encryption scheme version, field can be omitted, default is \\\verb#AES/GCM/NoPadding#, 256-bit key.
AES Galois Counter Mode is a preferred way of encryption. 
It is authenticated encryption (simultaneously encrypts and authenticates the plaintext data). In GCM mode, no padding is used (not a bug).

\item \verb#esymBlock# contains symmetrically encrypted \verb#messagePayload#. Initialization vector is concatenated with the ciphertext.\\
$esymBlock = IV || ENCRYPT(iv, encKey01, messagePayload)$

\item \verb#hmacVersion# contains hmac scheme used, field can be omitted, default is \verb#HmacSHA1#.

\item \verb#hmac# HMAC on the message fields. Protects same fields as signature + signature field, 
asymmetric encryption block and symmetric encryption block. Signature protects plaintext version of the message 
while HMAC protects whole message (ciphertext version). Helps to avoid chosen ciphertext attack. 
HMAC is easy to generate \& verify, has a small footprint while added security benefit is non-trivial.
Signature protects plaintext version of the message.

Note that it is a symmetric technique so attacker can still create packets with valid HMAC. Digital signature 
has to be verified (mac key is signed as well).

\begin{itemize}
 \item \verb#protocolType#
 \item \verb#protocolVersion#
 \item \verb#ampType#
 \item \verb#ampVersion#
 \item \verb#sequenceNumber#
 \item \verb#randomNonce#
 \item \verb#sender#
 \item \verb#destination#
 \item \verb#destinationType#
 \item \verb#iv#, initialization vector for symmetric encryption.
 \item \verb#encKey01#, symmetric encryption key for the symmetric encryption block.
 \item \verb#macKey01#, hmac symmetric key.
 \item \verb#easymBlock#
 \item \verb#easymBlockVersion#
 \item \verb#esymBlock#
 \item \verb#esymBlockVersion#
\end{itemize}
  
$
hmac = HMAC(macKey01, protocolType:protocolVersion:ampType:ampVersion:sequenceNumber:randomNonce:sender:
destination:destinationType:iv:encKey01:macKey01:easymBlock:easymBlockVersion:esymBlock:esymBlockVersion)) 
$

\textbf{Update v2}: hmac now includes \verb#messageSentMiliUTC#;

\item \verb#encKey# [VIRTUAL] is symmetric encryption key extracted from asymmetric encryption block. 

\item \verb#macKey# [VIRTUAL] is symmetric HMAC key extracted from asymmetric encryption block. 
 
\item \verb#payload# [VIRTUAL] is decrypted symmetric block.

\item \verb#protocolType# [VIRTUAL] is protocol type - taken from lower protocol. It is a type of this protocol.

\item \verb#protocolVersion# [VIRTUAL] is protocol version - taken from lower protocol. It is a type of this protocol.

\end{itemize}

\paragraph{Virtual message fields.} Fields marked as \verb#[VIRTUAL]# are helper fields used only in offline phase, 
during message creation or message decryption. It helps to generate/verify a digital signature and HMAC on additional fields in a following way: these 
fields are filled in, signature and/or HMAC is generated/verified, virtual fields are erased and signature and HMAC is added to the message.
Virtual messages must not be filled in during message transfer. 

\paragraph{Version number discussion.} There are several ways how to encapsulate two different protocols.
Internet Protocol (IP) has version defined inside, while type of the protocol is defined in an underlying layer.
We decided to add version number to the underlying layer so there are no further requirements on the upper message format.
E.g., parser does not have to assume that the message is still ProtocolBuffers encoded with version number as a first field
in the message, so the different version of the protocol message can change its format significantly. Note that
this affects only a message dispatcher handling particular message type.

\subsection{STPSimpleGroup}
This section proposes an extension for the STPSimple protocol for group messaging.
Idea: instead of flat key structure, concept of \emph{keyslots} is proposed.

The AMP payload is encrypted with master symmetric encryption key, MK.
For each recipient the STP message contains one key slot. Each key slot encrypts MK with 
public key of the recipient. 

Benefit: single message can be sent to multiple recipients (e.g., by the server) without modification. Some
broadcasting feature of the underlying transport protocol can be used to optimize group messages in bandwidth constrained environments.

It can be used for group messaging or for one user account enabled on different user devices (multiple instances),
for each instance, a different key/certificate is used. Assumption: instances are distinguishable from
each other by some ID/pseudonym, as in XMPP case.

This is proposed as a separate protocol in order to simplify protocol design and use of the protocol.
In many cases the simple version suffices.

\subsection{STPSimple Authenticated (v1.0)}
STPSimpleAuth is a variant of STPSimple that does perform encryption, only authentication of transported payload. It may be useful for some message types (ACKs).
Both protocols can share the same underlying protobuf structure (as STPSimpleAuth uses only subset of fields defined in STPSimple). Here are relevant fields:

\begin{Verbatim}[frame=single]
message STPSimple {
  optional int32 ampType=1;
  optional int32 ampVersion=2;g.
  optional uint32 sequenceNumber=3;
  optional uint64 messageSentMiliUTC=4;
  optional uint32 randomNonce=5;
  
  optional string sender=6;
  optional string destination=7; 
  optional int32 destinationType=8;   // OPT. 
  
  optional bytes signature=9;
  optional int32 signatureVersion=10;   // OPT.
  
  optional bytes payload=21;
}
\end{Verbatim} 

\paragraph{Fields discussion.} 
\begin{itemize}
 \item \verb#payload# cointains plaintext application message (serialized to bytes)

 \item \verb#signatureVersion# contains signature scheme used for the digital signature. This field can be omitted, default is RSA signature.
 
 \item \verb#signature# is a RSA digital signature protects key message fields and the plaintext message. Signing plaintext 
 is intended to bind plaintext message to the sender identity. Signature binds following fields together:
\begin{itemize}
 \item \verb#protocolType#
 \item \verb#protocolVersion#
 \item \verb#ampType#
 \item \verb#ampVersion#
 \item \verb#sequenceNumber#
 \item \verb#messageSentMiliUTC# 
 \item \verb#randomNonce#
 \item \verb#sender#
 \item \verb#destination#
 \item \verb#destinationType#
 \item \verb#payload#
\end{itemize}

Other fields have the same meaning as for STPSimple.
\end{itemize}


\subsection{Further variants of STPs}
There may be different multiple STPs suitable for different purposes.
\begin{itemize}
 \item FileTransfer protocol. We may use our FT protocol already employed in the PhoneX architecture. It provides Perfect Forward Secrecy
 and has good anti-SPAM properties. 
 
 \item OTR\footnote{\url{https://otr.cypherpunks.ca/Protocol-v3-4.0.0.html}} protocol. It provides Perfect Forward Secrecy and is 
 de-facto standard for a secure communication in instant messages. Benefit is that the protocol is well-tested by community
 and provides \emph{deniable encryption}. Disadvantage is that initial authenticated key exchange takes 2 round trips what can be 
 problematic in some network limited setups. \\
 Proposed trade-off: Perform re-keying / authenticated key exchange once per time window (e.g., once a week), when both clients 
 are connected to WiFi or other wide bandwidth network. 
 
 \item Symmetric encryption / authentication protocol. When using OTR, both sides posses shared secret symmetric keys that can be used
 for symmetric encryption and HMAC. This message has smaller footprint since digital signature and asymmetric encryption blocks
 are not used. Moreover, it is computationally effective and does not provide non-repudiation, which is sometimes undesirable.
 
 \item Plain STP. For some message types (ACKs), encryption is not necessary.
\end{itemize}


\subsection{AMPSimple, v1.0}
This section briefly describes application message protocol for instant messaging. This messaging protocol is very simple,
it contains one field with text message from the user and another 32b int field nonce as a message identificator.

\begin{Verbatim}[frame=single]
message STPSimple {
  // Message sent by user. message=GZIP(plainMessage);
  optional bytes message=1;
  optional int32 nonce=2;
}
\end{Verbatim} 

Field message contains a GZIPed version of a plaintext message entered by user in UI. GZIP compression reduces 
overall message size.

\subsection{Versioning}

\subsubsection{Capabilities}
Capabilities are flags (version-like numbers) indicating what type of functionality remote client supports. Each client transfers list of capabilities to every other contact in his contact list. They are transferred using XMPP Presence.\\ \\
Messaging capability has format $3.1.x$, where $x$ represents Messaging Protocol version, 3.1 is messaging prefix. Following versions are supported:

\begin{itemize}
\item \textbf{3.1.2} - Messaging Protocol 2.0;
\item \textbf{3.1.2.1} - Messaging Protocol 2.1 (STPSimple v2 support);
\item default\footnote{When no messaging capability is indicated} - Messaging Protocol 1.0;

\end{itemize}





\section{Message processing - Android}
Here is a summary how Message processing on Android works/should work.

\begin{itemize}
\item \textbf{Message Queue} -  FIFO containing QueuedMessage(s) that have to be processed by MessageManager. Currently this queue is stored in SQLCipher and accessible by standard content resolver provided by Android.

\item \textbf{QueuedMessage} - item stored in Message Queue. Format:
\begin{Verbatim}[frame=single]
public class QueuedMessage {
    protected Integer id;
    protected long time;
    protected String from;
    protected String to;
    protected boolean isOutgoing;
    protected boolean isProcessed = false;
    protected Integer sendCounter;

    protected Integer transportProtocolType;
    protected Integer transportProtocolVersion;

    protected Integer messageProtocolType;
    protected Integer messageProtocolVersion;

    protected String mimeType; // legacy

    protected byte[] transportPayload;
    protected byte[] envelopePayload;    
    
    protected Integer referencedId;
}
\end{Verbatim} 
\textbf{Fields discussion}:
\begin{itemize}
\item \verb#from# is identity of sender. Currently we expect that it is in SIP format (email).
\item \verb#to# is identity of receiver. 
\item \verb#isProcessed# is flag saying that message was processed and is waiting for acknowledgment (after that, it is deleted). 
\item \verb#sendCounter# is send counter for outgoing messages. We try to resend (if transfer failed) each outgoing message up to specific number of times. 
\item \verb#transportProtocolType#, \verb#transportProtocolVersion# are identifiers of what (secure) transport protocol is or should be used (should - for outgoing messages). All proposed transport protocols are listed in \ref{subsec:message_protocol_design}. Here are those that are currently used:

\begin{tabular}{lll}
Transport Protocol name & Type \# & Supported Versions \# \\
\hline
\verb#S/MIME# & 1 & 1 \\
\verb#STPSimple# & 2 & 1
\end{tabular}
\medskip

\item \verb#messageProtocolType#, \verb#messageProtocolVersion# are identifiers of what AMP (application message protocol) is used. Here are types and versions that are currently supported:

\begin{tabular}{lll}
AMP name & Type \# & Supported Versions \# \\
\hline
\verb#Text# & 1 & 1 (plaintext), 2 (\verb#AmpSimple#) \\
\verb#Notification# & 2 & 1 (\verb#GeneralMsgNotification#, see FileTransfer protocol)
\end{tabular}
\medskip

\item \verb#mimeType# is a legacy field (stores MIME type that is sent via SIP messaging system)
\item \verb#transportPayload# is byte payload for S/MIME transport protocol (in this case it's Base64 encoded String represented as bytes) or StpSimple transport layer.
\item \verb#envelopePayload# is byte payload of message envelope, see \ref{subsec:message_protocol_design}.
\item \verb#referencedId# is a reference to application item that was passed to MessageQueue for sending. For example, in case of text message, this is an id of item in SipMessage table. It is used for callbacks to application layer when informing about sending status.

\end{itemize}

\end{itemize}
\subsection{Application flow (incoming messages)}
This is the simplest flow case, because we do not have to solve message retransmission.

Received message is retrieved in \verb#PjCallback.on_pager()# method. This method takes message data, parses it using ProtoBuf as \verb#MessageEnvelope# object and stores envelope payload together with transport protocol type and version into MessageQueue. 

MessageQueue detects changes in associated DB table. In case there are some unprocessed incoming messages, those are passed to \verb#TransportProtocolDispatcher# together with user identity and remote certificate (that might be necessary when decrypting received transport packet). After this, QueuedMessage is deleted from the DB queue.

\verb#TransportProtocolDispatcher# takes transport packet and passes it to associated parser according to transport type and version. After packet is parsed and data is retrieved, this is further passed to \verb#AmpDispatcher#.

\verb#AmpDispatcher# processes message accordingly to \verb#AmpType# and \verb#AmpVersion#.


\subsection{Application flow (outgoing messages)}

If application want to send particular message over the wire, it needs to put it in the MessageQueue first. There are few static methods provided in \verb#AmpDispatcher# to do so:

\begin{itemize}
\item a
\end{itemize}


\subsubsection{Manual resend}
Manual resend works almost the same as regular sending. Difference from automatic resend is that we do not want to reuse transport protocol envelope but generate a new one (e.g. to get new transport nonce). Also, we need to notify MessageManager about resend so it can correctly update MessageQueue. Options: 
\begin{enumerate}
\item \textbf{Update old QueuedMessage} - reset \verb#sendCounter#, delete \verb#envelopePayload#, \verb#finalMessage# and \linebreak \verb#finalMessageHash#. Mark message as unprocessed. This preserves message's position in the queue. \\
\item \textbf{Delete old QueuedMessage + insert new} - delete old message according to \verb#referencedId# if any such message is in the MessageQueue (we can do this before every sending in AmpDispatcher). Message is reinserted in the queue.
\end{enumerate}

Currently the \textbf{2nd option} is preferred and used in the Android application.

\subsubsection{Timers}
Situation: message is flagged as 'processed', but no acknowledgment is received it is blocking other messages in the stack. 
Reality: In case of sudden connection lost, underlying SipStack provides negative ack, so no timer is necessary.
Situation: application crashes before receving any ack - so message is blocking.
Solution

\end{document}
